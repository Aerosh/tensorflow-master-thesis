\subsection{Variants}

\subsubsection{Alternative norms}

The following norms can be used instead of the $\ell_2$ norm to define $\drlx_i$ and  $\srlx_i$ in \eqref{eq:drlx} and \eqref{eq:srlx}. 

\paragraph{Group-normalized $\ell_2$ norm.} The group-normalized $\ell_2$ norm, where $|\ms G|$ denotes the cardinality of group $\ms G$ is given by
\begin{equation}
  \normg{\ms G}{\mv k} \triangleq \left( \frac{1}{|\ms G|}\sum_{\ms I \in \ms G} | k_{\ms I} |_2 \right )^{\frac{1}{2}}.
\end{equation}
It has the property that, assuming $\E |k_{\ms I}|_2^2 = C^2$ for active groups (otherwise $\mv k(\ms G) = \mv 0$ ) \footnote{Not valid?...}
\begin{equation} \cite{eq:assumption}
  \E \left [ \left (\normg{\ms G}{\mv k} \right )^2 \right ] =  C^2
\end{equation}
does not depend on the group size. Accordingly, from Cauchy-Schwartz, 
\begin{equation}
  \E \left [\normg{\ms G}{\mv k} \right ] \leq   \sqrt{  \E \left [ \left ( \normg{\ms G}{\mv k} \right )^2 \right ] } = C
\end{equation}
implies that the expected norm is bounded for any group size. 

The potential benefit of using $\normg{\ms G}{\mv k}$ to define, for example $\drlx_i$ in \eqref{eq:drlx} is that each term corresponding to an active group in the summation will be bounded by $C$, and hence
\begin{equation}
\drlx_i \leq d_i \cdot C,
\end{equation}
where $d_i$ is the learned width for layer $i$.

\paragraph{Group-scaled $\ell_2$ norm.}
The work in \cite{} and \cite{} proposes the opposite approach 
\begin{equation}
  \normg{\ms G}{\mv k} \triangleq \left( {|\ms G|}\sum_{\ms I \in \ms G} | k_{\ms I} |_2 \right )^{\frac{1}{2}}.
\end{equation}
The motivation used in \cite{} is that, by using this normalization, the resulting problem produces the ANOVA solution.

\paragraph{Other approaches.}
It is also possible to use other norms instead of the $\ell_2$ norm, for example, the $\ell_\infty$ has been proposed elsewhere.

\paragraph{Batch-normalization variant.}
One potential problem with the above analysis is that the assumption in \eqref{eq:assumption} is not valid. A possible reason for this is that the dynamic range of the layer-input activations can vary widely from layer to layer. The batch normalization technique \cite{} addresses this problem by scaling and shifting the activation coefficients $\m M_i$ produced by a given layer before feeding them to the next layer. This normalization is done in two steps on a per-feature-map basis. Let $\mv m$ denote a feature map of $\mv M$ and $\mu$ and $\sigma$ denote an estimate of the mean and variance of the entries $\mv m$ derived from the current training batch, the first step 
\begin{equation}
\mv m^\prime = \frac{1}{\sigma}( \mv m - \mv 1 \mu )
\end{equation}
ensures that the resulting feature maps (over the batch) have unit variance and zero mean.

The next step gives the system back the liberty to learn an adequate mean and variance by means of the learned parameters $a$ and $b$:
\begin{equation}
\mv m'' = a \mv m' - b
\end{equation}

\subsection{Complexity regularizers.}
The complexity expression \eqref{eq:complexity} has a square depdendence on both the kernel size and the layer width that is not reflected in the regularizers in \eqref{eq:reg1} and \eqref{eq:reg2}. This suggests instead using
\begin{equation}\label{eq:reg1 sq}
\Omega_1(\mm \Theta) = \sum_{i=1}^D  {\drlx_i}^2 % \sum_{j=1}^{\dmax_i} |\mv k_{ij}|_2.
\end{equation}
 and 
\begin{equation}\label{eq:reg2 sq}
  \Omega_2(\mm \Theta) = \sum_{i=1}^D\sum_{j=1}^{\dmax_i} {{\srlx_{ij}}}^2.
\end{equation}

Alternatively, one can substitue $d_i$ and $s_{ij}$ in \eqref{eq:complexity} by their relaxed versions to obtain
\begin{equation}
  \Omega_3(\mm \Theta) = \sum_{i=1}^D \sum_{j=1}^{d_i} r_i \cdot c_i \cdot \drlx_{i-1} \cdot {\srlx_{ij}}^2.
\end{equation}


%%% Local Variables:
%%% mode: latex
%%% TeX-master: "main"
%%% End:
