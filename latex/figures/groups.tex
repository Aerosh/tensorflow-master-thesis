\def\size{5}
\def\sep{0.2cm}
\def\xwidth{.38}

\tikzset{axis0/.style={/pgfplots/xlabel=$q$, /pgfplots/ylabel=$p$, /pgfplots/ylabel style={rotate=-90}}, /pgfplots/title style={inner sep=0, outer sep=0}}
\tikzset{axis1/.style={}}
\tikzset{axis2/.style={}}
\tikzset{axis3/.style={}}
\tikzset{axis4/.style={}}
\tikzset{axis5/.style={}}

\begin{figure*}
  \centering
  \begin{tikzpicture}[node distance=0cm]
    \coordinate (origin) at (0,0);    
    \foreach \top/\bot/\left/\name/\group in {1/0/origin/axis0/1,1/1/axis0/axis1/2,2/1/axis1/axis2/3,2/2/axis2/axis3/4,3/2/axis3/axis4/5}{
        \begin{axis}[
          title=$\ms G_\group$,
          \name,
          name=\name, at={($(\left.east)+(\sep,0)$)}, anchor=west,
          width=\xwidth\linewidth,
          xmin=0, ymin=0, xmax=\size, ymax=\size, 
          axis on top,
          ticks=none,
          xtick = {0,...,\size},
          ytick = {0,...,\size},
          unit vector ratio=1 1 1,
          ]      
          % Draw all half rings
          \drawsquare{0}{0}{\size}{cyan}
          \drawsquare{1}{0}{\size}{white}%
          \drawsquare{1}{0}{\size}{blue}
          \drawsquare{1}{1}{\size}{white}%
          \drawsquare{1}{1}{\size}{violet}
          \drawsquare{2}{1}{\size}{white}%
          \drawsquare{2}{1}{\size}{purple}
          \drawsquare{2}{2}{\size}{white}%
          \drawsquare{2}{2}{\size}{orange}

          % Hide one part
          \drawsquare{\top}{\bot}{\size}{white}
        \end{axis}
      }
    \end{tikzpicture}
    \begin{tikzpicture}
      \node[minimum size=1cm]{};
    \end{tikzpicture}
    \begin{tikzpicture}[node distance=0cm, ]
      \coordinate (origin) at (0,0);
      \foreach \top/\bot/\left/\name/\group in {1/1/origin/axis0/1,2/2/axis0/axis1/2,2/3/axis1/axis2/3}{
          \begin{axis}[
            \name,
            title=$\ms G_\group$,
            name=\name, at={($(\left.east)+(\sep,0)$)},
            anchor=west,
            width=\xwidth\linewidth,
            xmin=0, ymin=0, xmax=\size, ymax=\size, 
            axis on top,
            ticks=none,
            xtick = {0,...,\size},
            ytick = {0,...,\size},
            unit vector ratio=1 1 1,
            ]      
            % Draw all half rings
            \drawsquare{0}{0}{\size}{cyan}
            \drawsquare{1}{1}{\size}{white}%
            \drawsquare{1}{1}{\size}{blue}
            \drawsquare{2}{2}{\size}{white}%
            \drawsquare{2}{2}{\size}{violet}        
            
            % Hide one part
            \drawsquare{\top}{\bot}{\size}{white}
          \end{axis}
        }

        \end{tikzpicture}  
        \caption{Examples of supports (positions of colored entries) used for groups $\ms G_n$ for the case $\smax=5$ using \textit{(left)} a \emph{nested half-ring} strategy with $\smax$ groups resulting in $s_{ij} \in \{0,1,2,\ldots,\smax_i\}$; and \textit{(right)} a \emph{nested ring} strategy with $\lceil \frac{\smax}{2} \rceil$ groups resulting in $s_{ij} \in \{0,1,3, \ldots, \smax_i\}$ that are zero or odd. For a given kernel $\mv k_{ij}$ with entries $k_{pqr}$, $1 \leq p,q \leq \smax_i$, $1 \leq r \leq \dmax_{i-1}$ , each group $\ms G_n$ contains all triplets $(p,q,r)$ corresponding to the shaded area. Note that the shaded supports only depend on the spatial indices $p$ and $q$ -- for a given group, the same support is used for all of the kernel's channels.}
        \label{fig:groups}
      \end{figure*}

      %   %%% Local Variables:
      %   %%% mode: latex
      %   %%% TeX-master: "../main"
      %   %%% End:
