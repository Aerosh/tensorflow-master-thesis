\section{Introduction}

depth, kernel size, layer width

\paragraph{Notation:}
We denote scalars using standard typeface (\eg scalar $a$), and we denote vectors, matrices and three-dimensional tensors using bold typeface (\eg a tensor $\mv M  \in \R^{a \times b \times c}$). Given a vector, tensor or matrix $\mv M$, we define its $\ell_p$ norm for all $p>0$ as $|\mv M|_p \triangleq \left ( \sum_{m \in \mm M} |m|^p \right )^{\frac{1}{p}}$; we also use $|\mv M|_0$ to denote the $\ell_0$ pseudo-norm which produces the number of non-zero elements in $\mv M$ (equal to $|\mv M|_p$ as $p\rightarrow 0$).\footnote{Strictly speaking, $|\mv M|_p$ is a norm only for $p\geqslant 1$. } %For ease of \PP{TBC?}

%We use $\mv v_k$ to denote a vector from a sequence $\mv v_1, \mv v_2, \ldots, \mv v_N$, and $v_k$ to denote the $k$-th coefficient of vector $\mv v$. We let $[\mv a_k]_k$ (respectively, $[a_k]_k$) denotes concatenation of the vectors $\mv a_k$ (scalars $a_k$) to form a single column vector. Finally, we use $\Jac{\mv y}{\mv x}$ to denote the Jacobian matrix with $(i,j)$-th entry $\Jac{y_i}{x_j}$.




%%% Local Variables:
%%% mode: latex
%%% TeX-master: "main"
%%% End:
